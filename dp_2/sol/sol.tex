\documentclass[hyperref={unicode=true}]{beamer}
\usepackage{multirow}
\usepackage{minted}
\usepackage{amsmath} % used for boldsymbol.
\renewcommand{\vec}[1]{\boldsymbol{#1}} % Uncomment for BOLD vectors.
\usepackage[slantfont,boldfont]{xeCJK}
\setCJKmainfont{SimSun}
\setCJKmonofont{FangSong}
\setCJKsansfont{SimHei}
\usetheme{Darmstadt}
\usecolortheme{beaver}
\usepackage{booktabs}
\usepackage{tikz}
\setbeamertemplate{theorems}[numbered]
\newtheorem{mytht}{\bf 定理}
\theoremstyle{definition}
\newtheorem{mydef}[]{\bf 定义}
\theoremstyle{proof}
\newtheorem{myprf}[]{\bf 证明}
\usepackage{ulem}
\usepackage[T1]{fontenc}
\usepackage{inconsolata}
\makeatletter
% there's no italic/slanted for Inconsolata
\@namedef{T1/zi4/m/it}{<->ssub*zi4/m/n}
\@namedef{T1/zi4/b/it}{<->ssub*zi4/b/n}
\@namedef{T1/zi4/bx/it}{<->ssub*zi4/b/n}
\@namedef{T1/zi4/m/sl}{<->ssub*zi4/m/n}
\@namedef{T1/zi4/b/sl}{<->ssub*zi4/b/n}
\@namedef{T1/zi4/bx/sl}{<->ssub*zi4/b/n}
\makeatother

\input zhwinfonts
\begin{document}
\setbeamertemplate{caption}[numbered]
\renewcommand\figurename{图}
\renewcommand\tablename{表}
\renewcommand\contentsname{\centering 目录}


%%------------------------------------------
\title{动态规划进阶(二)练习题题解}
\subtitle{}
\author{马玉坤}
\institute{哈尔滨工业大学计算机科学与技术学院}
\date{2017年8月18日}
%%------------------------------------------

\begin{frame}\titlepage\end{frame}

\begin{frame}{Magic Trick}
  \framesubtitle{POJ 2282}
  题意:给你一个区间,求这个区间内各个数位上,数字0-9分别出现的次数。\\
  设$dp(i,j,k,l)$表示枚举到第i位的满足题意的数目的个数,j表示之前确定的高位是否已达到上限,k表示高位是否全为0,l表示统计的是0-9哪个数字。\\
  减少转移的做法:当j=0 (高位未达到上限)时,第i位已经更低位可以随便填,这时候0-9是均匀分布的。
\end{frame}

\begin{frame}{Round Numbers}
  \framesubtitle{POJ 3252}
  题意:求区间$[L,R]$中所有满足二进制表示中1的个数等于0的个数的数的个数。\\
  设$f(i,j,k,l)$表示枚举到第i位,j表示高位是否达到限制,k表示高位是否全为0(该题在统计时不能加入签到零),l等于已确定的位中1的个数与0的个数的差。
\end{frame}

\begin{frame}{windy数}
  \framesubtitle{BZOJ 1026}
  题意:求区间$[L,R]$内有多少个windy数。windy数的定义为:不含签到零且相邻两个数字之差至少为2的正整数。\\
  设$f(i,j,k,l)$表示枚举到第i位,j表示高位是否达到限制,k表示高位是否全为0(该题在统计时不能加入签到零),l表示第i位填的数。
\end{frame}


\begin{frame}{Just Too Lucky}
  \framesubtitle{NEERC 2008 J}
  题意:求小于等于n的所有满足能整除各位数之和的数。比如$42 \mod 6 = 0$。\\
  其实各位数之和最小为1,最大也就到$9 \times 12 = 108$,所以可以枚举各位数之和(设各位数之和为$j$),然后dp求各位数之和为j且能整除j的数的个数。\\
  设$f (i,k,mod)$表示枚举到第i位,已经确定的位数对刚才枚举的j取模后的结果为mod的数字的个数,k表示高位是否达到上限。(本题目可以添加前导零。)
\end{frame}

\begin{frame}{Periodical Numbers}
  \framesubtitle{Codeforces 215 E}
  题意:求区间$[L,R]$内的周期二进制数个数。周期二进制数指的是能找到循环周期的二进制数,比如01110111。\\
  枚举位数L,然后枚举周期长度P(周期长度需要满足整除位数),然后找小于等于n的位数为L,周期长度为P的二进制数个数。\\
  设$f(i,j)$。i表示枚举到第i位,j表示高位是否达到上限。该题有前导零限制,但是由于我们手动枚举位数(也就是第一个1的位置),所以不需设置前导零标记。\\
  注意,在dp过程中,我们需要检查是否违反了低位的限制。比如n=111000,如果我们高三位已经确定为111,这时候是不合法的;当然高三位如果是110,这时候就是合法的。
\end{frame}

\begin{frame}{Painter's Problem}
  \framesubtitle{POJ 1681}
  题意:给定一个矩阵,矩阵上每个格子有初始的颜色(黄色或者白色)。没次操作可以使一个格子以及周围4个格子反转颜色(黄变白,白变黄),问最少需要多少此操作,能让所有格子变成黄色。\\
  枚举第一行的格子反不反转,就知道第二行该怎么做,然后就知道第i行该怎么做。\\
  换句话说,枚举完第一行的决策,剩余行的决策就都确定了。
\end{frame}

\begin{frame}{Corn Fields}
  \framesubtitle{POJ 3254}
  题意:在一片M行N列的草地上(用0和1矩阵表示),1表示能放牛,0表示不能放。在草地上放牛并且牛不能相邻,问有多少种放法(一头牛都不放也算一种)。\\
  设$f(i,S)$表示从第一行枚举到第i行,且第i行状态为S时的方案数。\\
  假设第i行状态S的第j位为1((i,j)放牛),那么第i-1行的第j为就必须为0,而第i行第j列为0,那么第i-1行第j列既可以为0,也可以为1。\\
  直接状压压力不大,有兴趣可以搜索“高位前缀和”(本质也是状压)。
\end{frame}

\begin{frame}{Rectangular Covering}
  \framesubtitle{POJ 2836}
  平面上有n个点,现用平行于坐标轴的矩形去覆盖所有点,每个矩形至少盖两个点,矩形面积不可为0,求这些矩形的最小面积和。\\
  因为面积最小化,所以每个矩形一定至少有两个端点与n个点中的两个点分别重合。\\
  求出以每对点为端点形成的矩形所包含的点集($\frac{n(n-1)}{2}$对点)。
  设$f(S)$表示覆盖的点集为S所需要的最小矩形面积,可以枚举这$n(n-1)/2$个矩形(设枚举到第i个矩形),用$f(S)+area(i)$去更新$f(S|cover(i))$。
\end{frame}

\begin{frame}{Mondriaan's Dream}
  \framesubtitle{POJ 2411}
  多米诺骨牌覆盖问题,见上午课件。
\end{frame}

\begin{frame}{Sherlock and the Encrypted Data}
  \framesubtitle{Codeforces 776 G}
  题意:求区间$[L,R]$内所有满足。。。题意挺麻烦的,大家直接看题目吧。\\
  一个数异或另一个数变大还是变小,只跟另一个数的最高位的1有关。所以为了判断x是否是满足题意的数字,我们只需要关注x的16进制表示中数字最大的一位数(设这个数为i,$0 \leq i < 16$)以及x的二进制表示中的对应位(就是第i位)就可以。\\
  所以我们可以枚举h,表示十六进制中表示中各位上最大的数为h。设方程$f(i,j,k,l)$。i表示枚举到第i位,j表示高位是否达到上限,k表示目前为止枚举的十六进制位中有没有出现过h,l表示前导0。需要注意的是,每一个十六进制位都需要小于等于h,而且二进制表示中第h位应该等于1。(这样异或$2^h$之后才能减小。)\\
  官方题解用了{\bf 不同的做法},把数位DP与状压DP结合了起来,大家也可以做一下参考。\\
  \url{http://codeforces.com/blog/entry/50622}
\end{frame}
\end{document}
